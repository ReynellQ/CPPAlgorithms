\documentclass{article}
\usepackage[utf8]{inputenc}
\usepackage{amsmath}
\usepackage{natbib}
\usepackage{graphicx}
\usepackage{listings}
\usepackage{xcolor}

\definecolor{colorGreen}{rgb}{0,0.6,0}
\definecolor{colorGray}{rgb}{0.5,0.5,0.5}
\definecolor{codepurple}{rgb}{0.58,0,0.82}
\definecolor{backcolour}{rgb}{0.95,0.95,0.92}

\lstdefinestyle{mystyle}{
    backgroundcolor=\color{backcolour},   
    commentstyle=\color{colorGreen},
    keywordstyle=\color{blue},
    numberstyle=\tiny\color{colorGray},
    stringstyle=\color{orange},
    basicstyle=\ttfamily\footnotesize,
    breakatwhitespace=false,         
    breaklines=false,                 
    captionpos=b,                    
    keepspaces=true,                 
    numbers=left,                    
    numbersep=5pt,                  
    showspaces=false,   
    showstringspaces=false,
    showtabs=false,                  
    tabsize=2
}

\lstset{style=mystyle}


\title{Handbook}


\begin{document}

\maketitle
\tableofcontents
\pagebreak
\section{Grafos}
    \subsection{Dinic}
        Insertar utilidad del algoritmo:
        \\ \\
        C++:
        \lstinputlisting[language=C++]{C++/Dinic.cpp}

\pagebreak
\section{Strings}
    \subsection{Función Z}
        Insertar utilidad del algoritmo:
        \\ \\
        C++:
        \lstinputlisting[language=C++]{C++/Strings/FuncionZ.cpp}
    \subsection{KMP}
        Insertar utilidad del algoritmo:
        \\ \\
        C++:
        \lstinputlisting[language=C++]{C++/Strings/kmp.cpp}
    \subsection{Suffix array}
        Devuelve un arreglo con el orden lexicográfico de los sufijos de un string S
        \\ \\
        C++:
        \lstinputlisting[language=C++]{C++/Strings/Suffix Array/SuffixArray.cpp}
        Java:
        \lstinputlisting[language=Java]{Java/Strings/Suffix Array/SuffixArray.java}
    \subsection{Longest Common Prefix on Suffixs}
        Devuelve un arreglo que contiene el largo del prefijo común máximo entre 2 sufijos i e i+1
        \\ \\
        C++:
        \lstinputlisting[language=C++]{C++/Strings/Suffix Array/LCP.cpp}
        Java:
        \lstinputlisting[language=Java]{Java/Strings/Suffix Array/LCP.java}
\pagebreak
\section{Búsqueda}
    \subsection{Ternary Search}
        Insertar utilidad del algoritmo:
        \\ \\
        C++:
        \lstinputlisting[language=C++]{C++/ternary_search.cpp}
\pagebreak
\section{Geometría}
    \subsection{Convex Hull}
        Insertar utilidad del algoritmo:
        \\ \\
        C++:
        \lstinputlisting[language=C++]{C++/convex_hull.cpp}
\pagebreak
\section{Matemáticas}
    \subsection{Factorial modulo m}
        Permite calcular $n! \mod{m}$
        \\ \\
        C++:
        \lstinputlisting[language=C++]{C++/Matematicas/factorial.cpp}
    \subsection{Exponenciacion binaria}
        Permite calcular $c \equiv a^b\pmod{m}$
        \\ \\
        C++:
        \lstinputlisting[language=C++]{C++/Matematicas/bin_pow.cpp}

    \subsection{Inverso Modular}
        Permite calcular $ a^{-1} \mod{m}$, este número satisface $a \cdot a^{-1} \equiv 1 \pmod{m}$
        \\ \\
        Con el pequeño teorema de Fermat, siempre que m sea primo, se calcula $x \equiv a^{m - 2} \pmod{m}$, siendo x su inverso modular.
        \\ \\
        Con el algoritmo de Euclides extendido, siempre y cuando $gcd(a, m) = 1$, se calculan $x, y$ tal que $ ax + my = 1$, por lo que
        $ax \equiv 1 \pmod{m}$, siendo x el inverso modular 
        \\ \\
        C++:
        \lstinputlisting[language=C++]{C++/Matematicas/Inverso Modular.cpp}
    \subsection{Inverso modular del factorial modulo m}
        Permite calcular $ i!^{-1} \mod{m}$ para todo $1 \leq i \leq N$
        \\ \\
        C++:
        \lstinputlisting[language=C++]{C++/Matematicas/factorial.cpp}
    \subsection{Coeficientes binomiales modulo m}
        Calculo de $\binom{n}{k} \mod{m}$ de múltiples formas
        \subsubsection{nCk $\mod{m}$ si m es primo}
            Para $m \geq 10^9$, se puede emplear la fórmula recursiva
            \[
                \binom{n}{k} = \binom{n - 1}{k - 1} + \binom{n - 1}{k} \mod{m}
            \]
            O la formula explicita mediante factoriales
            \[
                \binom{n}{k} = \frac{n!}{k!(n-k)!}  \mod m = n!\, k!^{-1}(n - k)!^{-1} \mod{m}
            \]
            \\ \\
            C++:
            \lstinputlisting[language=C++]{C++/Matematicas/combinatorial.cpp}
            
            Para $ m \leq 10^5$, se puede usar el teorema de Lucas que plantea
            \[
                \binom{n}{k} \mod{m} = \prod_{i = 1}^{log{m}}\binom{n_i}{k_i}
            \]
            Donde
            \[
                n_i = \frac{n_{i - 1}}{m}, \qquad n_0 = n
            \]
            \[
                k_i = \frac{k_{i - 1}}{m}, \qquad k_0 = k
            \]
        \subsubsection{nCk $\mod{m}$ si m es compuesto}
            Se realiza la descomposición en factores primos de m, resultando
            \[
                k_i = \frac{k_{i - 1}}{m}, \qquad k_0 = k
            \]
            Por cada factor primo se computa 
    \subsection{Miller-Rabin}
        Insertar utilidad del algoritmo:
        \\ \\
        C++:
        \lstinputlisting[language=C++]{C++/Miller-Rabin.cpp}
    \subsection{Pollard Rho}
        Insertar utilidad del algoritmo:
        \\ \\
        C++:
        \lstinputlisting[language=C++]{C++/Pollard Rho.cpp}
\pagebreak
\end{document}