\documentclass{article}
\usepackage[utf8]{inputenc}
\usepackage{amsmath}
\usepackage{natbib}
\usepackage{graphicx}
\usepackage{listings}
\usepackage{xcolor}

\definecolor{colorGreen}{rgb}{0,0.6,0}
\definecolor{colorGray}{rgb}{0.5,0.5,0.5}
\definecolor{codepurple}{rgb}{0.58,0,0.82}
\definecolor{backcolour}{rgb}{0.95,0.95,0.92}

\lstdefinestyle{mystyle}{
    backgroundcolor=\color{backcolour},   
    commentstyle=\color{colorGreen},
    keywordstyle=\color{blue},
    numberstyle=\tiny\color{colorGray},
    stringstyle=\color{orange},
    basicstyle=\ttfamily\footnotesize,
    breakatwhitespace=false,         
    breaklines=false,                 
    captionpos=b,                    
    keepspaces=true,                 
    numbers=left,                    
    numbersep=5pt,                  
    showspaces=false,   
    showstringspaces=false,
    showtabs=false,                  
    tabsize=2
}

\lstset{style=mystyle}


\title{Handbook}


\begin{document}

\maketitle
\tableofcontents
\pagebreak
\section{Grafos}
    \subsection{Dinic}
        Insertar utilidad del algoritmo:
        \\ \\
        C++:
        \lstinputlisting[language=C++]{C++/Dinic.cpp}

\pagebreak
\section{Strings}
    \subsection{Función Z}
        Insertar utilidad del algoritmo:
        \\ \\
        C++:
        \lstinputlisting[language=C++]{C++/Strings/FuncionZ.cpp}
    \subsection{KMP}
        Insertar utilidad del algoritmo:
        \\ \\
        C++:
        \lstinputlisting[language=C++]{C++/Strings/kmp.cpp}
    \subsection{Suffix array}
        Devuelve un arreglo con el orden lexicográfico de los sufijos de un string S
        \\ \\
        C++:
        \lstinputlisting[language=C++]{C++/Strings/Suffix Array/SuffixArray.cpp}
        Java:
        \lstinputlisting[language=Java]{Java/Strings/Suffix Array/SuffixArray.java}
    \subsection{Longest Common Prefix on Suffixs}
        Devuelve un arreglo que contiene el largo del prefijo común máximo entre 2 sufijos i e i+1
        \\ \\
        C++:
        \lstinputlisting[language=C++]{C++/Strings/Suffix Array/LCP.cpp}
        Java:
        \lstinputlisting[language=Java]{Java/Strings/Suffix Array/LCP.java}
\pagebreak
\section{Búsqueda}
    \subsection{Ternary Search}
        Insertar utilidad del algoritmo:
        \\ \\
        C++:
        \lstinputlisting[language=C++]{C++/ternary_search.cpp}
\pagebreak
\section{Geometría}
    \subsection{Convex Hull}
        Insertar utilidad del algoritmo:
        \\ \\
        C++:
        \lstinputlisting[language=C++]{C++/convex_hull.cpp}
\pagebreak
\section{Matemáticas}
    \subsection{Inverso Modular}
        Insertar utilidad del algoritmo:
        \\ \\
        C++:
        \lstinputlisting[language=C++]{C++/Inverso Modular.cpp}
    \subsection{Miller-Rabin}
        Insertar utilidad del algoritmo:
        \\ \\
        C++:
        \lstinputlisting[language=C++]{C++/Miller-Rabin.cpp}
    \subsection{Pollard Rho}
        Insertar utilidad del algoritmo:
        \\ \\
        C++:
        \lstinputlisting[language=C++]{C++/Pollard Rho.cpp}
\pagebreak
\end{document}